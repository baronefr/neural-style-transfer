% this dummy chapter requires lipsum

\chapter{Feature demo}

In this dummy chapter we will see some features of this document.


\section{Side annotations}
\lecture{1, 01/01/2022}

\lipsum[1]\marginnote{Notes can be written besides the text with \texttt{\textbackslash marginnote} command.}

\lipsum[2]\marginbox{This is another fancy way of creating a side annotation. Use \texttt{\textbackslash marginbox}.}

\lipsum[3]\marginboxRed{There are more colors to be used. Red, ...}

\lipsum[4]\marginboxCyan{... a lighter blue, ...}

\lipsum[5]\marginboxYellow{...and a very nice yellow.}

\lipsum[6]\\
The boxes can be places inside a text using the \texttt{inline} option. For instance, \marginboxCyan[inline]{this is a margin box, even though it is not placed in a margin...}
\lipsum[7]



Finally, you can 
\marginboxhl{highlight a piece of text}{IDK what to write here, lmao.}
and put a box next to it with \texttt{\textbackslash marginboxhl}.
\lipsum[8]

\marginbold{Margin bold}
\lipsum[9]

\margintt{Margin typewriter}
\lipsum[10]

\marginsf{Margin sf}
\lipsum[11]


\subsection{TODO annotations}

The \texttt{todo} annotations are placed in the margin and are meant to be removed when the document is finished. \textbf{All the todos are listed in the final page of this document}.

\lipsum[10]
\todowarning{Here's a TODO warning in the margin.}

\lipsum[11]
\todocritical{You can mark critical todos as well}

\lipsum[12]
\todoblue{and a blue}
\lipsum[13]

You can 
\todohl{highlight some specific text}{This should be revised later.} and put a note besides it.
\lipsum[14]
\todogreen{ohoh, we have also green here}




\section{Images}

You can add some missing figure placeholder using the \texttt{missingfigure} command.

\missingfigure[figwidth=6cm]{Testing a missing figure}

\lipsum[20]

\begin{figure}
\missingfigure{Testing another missing figure}
\caption{Hello there! This is a figure...}
\end{figure}

\lipsum[21]



\begin{wrapfigure}{r}{0.5\textwidth}
\missingfigure[figwidth=6cm]{Add an image \ldots}
\caption{Hello there!}
\end{wrapfigure}
Figures can be wrapped along a text in a \texttt{wrapfigure} environment. Look at \href{https://it.overleaf.com/learn/latex/Wrapping_text_around_figures}{this link} for more instructions.
\lipsum[31]

\marginpar{
\missingfigure[figwidth=0.95\marginparwidth]{}
\captionof{figure}{Figures can be placed in the margin too.}
}
\lipsum[32-33]



\section{Boxes}

The \texttt{tcolorbox} class allows the use of boxes. This template provides an automatic preset in \texttt{enbox} environment.

\begin{enbox}{Title of enbox}
\lipsum[30]
\end{enbox}




\section{Others}

\lipsum[40]

Package \texttt{booktabs} allows the creation of nicer tables.

\begin{table}[h!]
  \begin{center}
    \caption{Table using booktabs.}
    \label{tab:table1}
    \begin{tabular}{l|c|r}
      \toprule % <-- Toprule here
      \textbf{Value 1} & \textbf{Value 2} & \textbf{Value 3}\\
      $\alpha$ & $\beta$ & $\gamma$ \\
      \midrule % <-- Midrule here
      1 & 1110.1 & a\\
      2 & 10.1 & b\\
      3 & 23.113231 & c\\
      \bottomrule % <-- Bottomrule here
    \end{tabular}
  \end{center}
\end{table}

\lipsum[41]